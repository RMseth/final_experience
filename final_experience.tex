\documentclass[10pt,a4paper]{report}
\usepackage[letterpaper, portrait, margin=1in]{geometry}
\usepackage[latin1]{inputenc}
\usepackage{amsmath}
\usepackage{amsfonts}
\usepackage{amssymb}
\usepackage{graphicx}
\DeclareRobustCommand{\stirling}{\genfrac\{\}{0pt}{}}
\author{Seth Bertlshofer\\Alexis Tyler\\Dustin Ginos\\Kevin Burgon}
\title{Final Experience}
\graphicspath{{img/}}
\begin{document}
	\maketitle
% ######## sub 1 ########
	\textbf{Sub-Experience One: The Mad Mail Carrier}\\
	Suppose Sue is a Mail Carrier who is crazy.  He likes to ensure that non of the $n$ houses on his delivery route get the mail they are supposed to.  Your goal, should you choose to accept it, for this sub-experience is to determine the number of ways Sue can deliver mail so that no one gets their mail in two ways.  One method is to use the Principle of Inclusion/Exclusion (PIE).  The other method is to use an exponential generating function to solve a recurrence which you'll develop.  Put $D_n$ equal to the number of ways Sue can distribute mail to $n$ houses so that none of them gets the correct mail.\\
	
	\textbf{Sub-Experience One, Part One: PIE Approach}\\
	Use the PIE to determine $D_n$.\\
	
	\textbf{Solution: }\\
	\newline
% ***** solution section *****


	\textbf{Sub-Experience Two, Part One: PIE Approach}\\
	The formula you obtain above should involve a truncated power series for $e^{-1}$.  Show that $D_n = \lfloor\frac{n!}{e}+\frac{1}{2}\rfloor, for\ n > 0$. (For $n=0$, the formula doesn't work: $D_0 = 1$, but the formula gives 0.)\\
	\newline
	\textbf{Solution: }\\
	\newline
% ***** solution section *****


	\textbf{Sub-Experience Three, Part One: PIE Approach}\\
	\newline
	\textbf{SE 1.3.1.} Prove the recurrence $D_n = (n-1)D_{n-1} + (n-1)D_{n-2},\ for\ n\geq 2,\ and\ D_0 = 1, D_1=0$.\\
	\textbf{SE 1.3.2.} Deduce, from the above recurrence $D_n = nD_{n-1}+(-1)^n,\ for\ n\geq 1,\ and\ D_0 = 1$.\\
	\textbf{SE 1.3.3.} Use an exponential generating function to solve the recurrence from part 1.3.2\\
	\newline
	\textbf{Solution: }\\
	\newline
% ***** solution section *****


% ######## sub 2 ########
	\textbf{Sub-Experience Two: Developing the EGF Lobe}\\
	Here are a few counting problems, the first of which is particularly easy to solve with an exponential generating function.  The answer is very simple and so it should have an elegant combinatorial proof.
	\begin{enumerate}
		\item Using exponential generating function, determine the number of $n$-digit quaternary sequences built from \{0,1,2,3\} with an even number of 0's and an odd number of 1's.
		\item $Not\ using$ exponential generating functions, please solve the counting problem above.
		\item Solve the recurrence from the pizza-cutting problem using exponential generating functions; that is, find a closed formula for $P(n)$ where, for $n\geq1, P(n) = P(n-1) + n$ with $P(0) = 1$.  Of course, you know the solution should be $P(n) = \frac{1}{2}n^2 + \frac{1}{2}n + 1$.
	\end{enumerate}
	\textbf{Solution: }\\
	\newline
% ***** solution section *****
	\textbf{2.1}\\
	Assuming that the sequence allows for leading 0's, we can have two ways of calculating "odd number of 1's":
	\[\binom{4}{1} = \binom{4}{3} = 4\]
	For an "even number of 0's":\\
	There are two cases that we'll need to consider.
	\begin{enumerate}
		\item There can be either zero 0's or 2 0's.  The first case will then have the other 3 digits be either 2 or 3.  We then have $2^3$ choices.  If there are 2 0's then we will have $\binom{3}{2}2$ choices.  You can choose where you want to place the 0's and then the other numbers will be either 2 or 3.\\
		Overall in this case we will have $4(2^3 + 2\binom{3}{2}) = 4(8 + 2 * 3) = 56$ possibilities.
		\item The only possibility is that no other digit will be 0, because the \# of zeros must be even.  That means that the remaining 2 choices is either 2 or 3.  This then gives us $4 * 2 = 8$ possibilities for this case.\\
	\end{enumerate}
	
	Because case 1 and 2 are mutually exclusive, we then have a total of $56 + 8 = 64$ possibilities for the sequence which is $4^3$.
	
	
	\textbf{2.2}\\
	\textbf{2.3}\\

% ######## sub 3 ########
	\textbf{Sub-Experience Three: Catalan Numbers}\\
	Roll up your sleeves and prepare to get down and dirty with generating functions.  The goal of the following sequence of problems is to prove that the sequence $(b_n)_n\geq0$ generated by the recurrence relation\\
	\[b_n = b_0b_{n-1} + b_1b_{n-2} + ... + b_{n-1}b_0\ with\ b_0 = 1\]
	gives the answer to the following counting questions, and that $b_n = \frac{1}{n+1}{2n}\choose{n}$.  These numbers are called the $Catalan\ Numbers$ and are nearly as ubiquitous as the Fibonacci Numbers.\\
	\newline
	How many binary trees on $n$ vertices are there?  Follow the steps below to produce a closed formula for $b_n$ using a generating function.\\
	\newline
	3.0 Denote by $b_n$ the number of binary trees on vertices.  Please prove that $b_n = b_0b_{n-1} + b_1b_{n-2} + ... + b_{n-1}b_0$ with $b_0 = 1$.\\
	3.1 Let $g(x) = \sum_{n=0}^{\infty}b_nx^n$ be the generating function for the sequence $b_n n\geq0$ of the number of binary trees on $n$ vertices.  Show that $g(x) = 1+x(g(x))^2$.\\
	3.2 Use the quadratic formula to show that $g(x) = 1-\frac{\sqrt{1-4x}}{2x}$. Note that you will have two options to use for $g(x)$, one of which you must rule out.\\
	3.3 Recall $The\ Binomial\ Theorem:\ For\ any\ real\ number\ r,\ (1+z)^r = \sum_{k=0}^{\infty}\binom{r}{k}z^k$, where 
	\[\binom{r}{k} = \left \lbrace \overset{\frac{r(r-1)(r-2)...(r-k+1)}{k!}}{0,},\ \overset{r\in\mathbb{R},\ 0\leq k\in\mathbb{Z}}{k<0} \right \rbrace \]
	Use the Binomial Theorem to obtain $g(x) = \frac{1}{2x}(1-\sum_{k=0}^{\infty}\binom{\frac{1}{2}}{k}(-4x)^k)$.\\
	3.4 Manipulate $g(x) = \frac{1}{2x}(1-\sum_{k=0}^{\infty}\binom{\frac{1}{2}}{k}(-4x)^k)$ into $g(x) = \sum_{n=0}^{\infty}\binom{\frac{1}{2}}{k}(-1)^n2^{2n+1}x^n$.  The reason for doing this is so that $x^n$'s coefficient can be read (remember that $b_n$ is defined to be$ [x^n]g(x)$.)  The following steps are suggested, but not required; follow them only if you have no idea of your own.  Begin with $g(x) = \frac{1}{2x}(1-\sum_{k=0}^{\infty}\binom{\frac{1}{2}}{k}(-4x)^k)$.\\
	\begin{itemize}
		\item Pull out the first term on the summation,  and simplify
		\item Substitute $n+1$ for k;  This will allow the summation to begin at zero.
		\item Pull the negative that resulted from the first manipulation in this list into the summation. 
		\item Bring $\frac{1}{2x}$ into the summation.\\
	\end{itemize}
	
	3.5 Verify that $\binom{\frac{1}{2}}{n+1} = \frac{1}{n+1}\binom{\frac{n-1}{2}}{n}(-1)^n\frac{1}{2}$\\
	3.6 Verify that $\binom{2n}{n}\frac{1}{2^{2n}} = \binom{\frac{n-1}{2}}{n}$.\\
	3.7 Finally, coerce $g(x)$ into $\sum_{n\geq0}^{}\binom{2n}{n}\frac{x^n}{n+1}$.\\
	\newline
	\textbf{Solution: }\\
	\newline
% ***** solution section *****
	\textbf{3.0}\\
	\begin{center}
		\begin{tabular}{|c||l|c|}
			\hline
			n&pattern&ways\\
			\hline
			\hline
			n=0&&1 way\\
			\hline
			n=1&()&1 way\\
			\hline
			n=2&()(),(())&2 ways\\
			\hline
			n=3&()()(),()(()),(())(),(()()),((()))&5 ways\\
			\hline
			n=4&()()()(),()()(()),()(())(),()(()()),()((())),&\\&(())()(),(())(()),(()())(),((()))(),(()()()),&\\&(()(())),((())()),((()())),(((())))&14 ways\\
			\hline		
		\end{tabular}
	\end{center}
	\[b_1 = b_0b_0\]\[b_2 = b_1b_0 + b_0b_1\]\[b_3 = b_2b_0 + b_1b_1 + b_0b_2\]\[b_4 = b_3b_0 + b_2b_1 + c_1c_2 + c_0c_3  \]
	Once we found that pattern and plugged it into $b_n = b_0b_{n-1} + b_1b_{n-2} + ... + b_{n-1}b_0$ with $b_0 = 1$, it gave us the correct values.\\
	\newline
	\textbf{3.1}\\
	If we let $g(x) = \sum_{n-0}^{\infty}b_nx^n$ for $(b_n)n\geq0$ then $g(x) = b_0 + b_1x + b_2x^2 + b_3x^3 + ... + b_nx^n + ...$ If we multiply $g(x)$ by itself to obtain $(g(x))^2$ we get $(g(x))^2 = b_0b_0 + (b_1b_0 + b_0b_1)x + (b_2b_0 + b_1b_1 + b_0b_2)x^2$ which we can see by looking at the coefficients are equal to the equations used to obtain the Catalan numbers.  Therefore:
	\[(g(x))^2 = b_1 + b_2 + b_3x^2 + b_4x^3 + ...\]
	we can convert this back to $g(x)$ if we multiply by x and add $b_0$, therefore:
	\[g(x) = b_0 + x(g(x))^2\]
	and we know that $b_0 = 1$ so $g(x) = 1 + x(g(x))^2$.\\
	\newline
	\textbf{3.2}\\
	If we let $g(x) = g$ then we can also write $g(x) = 1 + x(g(x))^2$ as $g = 1 + xg^2 \Rightarrow xg^2 - g + 1 = 0$ Using the quadratic formula we can solve $xg^2 - g + 1 = 0$, therefore we get: $g(x) = \frac{1\pm\sqrt{1^2 - 4(x)(1)}}{2x}$ which simplified is: $g(x) = \frac{1\pm\sqrt{1-4x}}{2x}$.  Now we have two options: $g(x) = \frac{1 + \sqrt{1-4x}}{2x}$ and $g(x) = \frac{1 - \sqrt{1-4x}}{2x}$ by applying L'Hopital's rule we get $\lim_{x\to0^+}\frac{1-\sqrt{1-4x}}{2x} = \lim_{x\to0^+}\frac{2(1-4x)^{-\frac{1}{2}}}{2} = 1$\\
	The positive square root doesn't evaluate to 1, therefore we will use $g(x) = \frac{1 - \sqrt{1-4x}}{2x}$\\
	\newline
	\textbf{3.3}\\
	Recalling the Binomial Theorem: for any real number $r,(1+z)^r = \sum_{k=0}^{\infty}\binom{r}{k}z^k$, where $z = -4x$ and $r= \frac{1}{2}$.  Plugging z and r into $\sum_{k=0}^{\infty}\binom{r}{k}z^k$ we obtain $(1 - 4x)^{\frac{1}{2}} = \sum_{k=0}^{\infty}\binom{\frac{1}{2}}{k}(-4x)^k$ and plugging this into $g(x)$ we get:
	\[g(x) = \frac{1}{2}x(1 - \sum_{k=0}^{\infty}\binom{\frac{1}{2}}{k}(-4x)^k) \]
	\textbf{3.4}\\
	
	\textbf{3.5}\\
	
	\textbf{3.6}\\
	To verify that $\binom{\frac{1}{2}}{n+1} = \frac{1}{n+1}\binom{n-\frac{1}{2}}{n}(-1)^n\frac{1}{2}$ let us compute the first 5 values for each expression:\\
% need to make the rows taller and the space between tables larger
	\begin{center}
		\begin{tabular}{|c|c|}
			\hline
			\multicolumn{2}{|c|}{$\binom{\frac{1}{2}}{n+1} = $} \\[.25cm]
			\hline
			n&output\\
			\hline
			1&$-\frac{1}{8}$\\[.25cm]
			\hline
			2&$\frac{1}{16}$\\[.25cm]
			\hline
			3&$-\frac{5}{128}$\\[.25cm]
			\hline
			4&$\frac{7}{256}$\\[.25cm]
			\hline
			5&$-\frac{21}{1024}$\\[.25cm]
			\hline
		\end{tabular}
		\hspace{2em}
		\begin{tabular}{|c|c|}
			\hline
			\multicolumn{2}{|c|}{$\frac{1}{n+1}\binom{n-\frac{1}{2}}{n}(-1)^n\frac{1}{2} = $} \\[.25cm]
			\hline
			n&output\\
			\hline
			1&$-\frac{1}{8}$\\[.25cm]
			\hline
			2&$\frac{1}{16}$\\[.25cm]
			\hline
			3&$-\frac{5}{128}$\\[.25cm]
			\hline
			4&$\frac{7}{256}$\\[.25cm]
			\hline
			5&$-\frac{21}{1024}$\\[.25cm]
			\hline
		\end{tabular}
	\end{center}
	
	
	Therefore $\binom{\frac{1}{2}}{n+1} = \frac{1}{n+1}\binom{n-\frac{1}{2}}{n}(-1)\frac{1}{2}$ if true.\\
	\newline
	\textbf{3.6}\\
	To verify that $\binom{2n}{n}\frac{1}{2^{2n}} = \binom{n-\frac{1}{2}}{n}$ we will do the same thing we did in 3.5 and compute the first 5 value for each expression.\\
	\begin{center}
		\begin{tabular}{|c|c|}
			\hline
			\multicolumn{2}{|c|}{$\binom{2n}{n}\frac{1}{2^{2n}} = $} \\[.25cm]
			\hline
			n&output\\
			\hline
			1&$-\frac{1}{8}$\\[.25cm]
			\hline
			2&$\frac{1}{16}$\\[.25cm]
			\hline
			3&$-\frac{5}{128}$\\[.25cm]
			\hline
			4&$\frac{7}{256}$\\[.25cm]
			\hline
			5&$-\frac{21}{1024}$\\[.25cm]
			\hline
		\end{tabular}
		\hspace{2em}
		\begin{tabular}{|c|c|}
			\hline
			\multicolumn{2}{|c|}{$\binom{n-\frac{1}{2}}{n} = $} \\[.25cm]
			\hline
			n&output\\
			\hline
			1&$-\frac{1}{8}$\\[.25cm]
			\hline
			2&$\frac{1}{16}$\\[.25cm]
			\hline
			3&$-\frac{5}{128}$\\[.25cm]
			\hline
			4&$\frac{7}{256}$\\[.25cm]
			\hline
			5&$-\frac{21}{1024}$\\[.25cm]
			\hline
		\end{tabular}
	\end{center}
	
	Therefore $\binom{2n}{n}\frac{1}{2^{2n}} = \binom{n-\frac{1}{2}}{n}$ is true.\\
	\newline
	\textbf{3.7}

% ######## sub 4 ########
	\textbf{Sub-Experience Four: Non-Standard Dice}\\
	A 6-sided die labeled with the integers 1,2,3,4,5,6 will be called a $standard\ die$.  The goal for this part of the Midterm Experience is to determine all ways to label a pair of dice with positive integers so that the probabilities of rolling the usual sums 2,3,...,12 are the same, but the labels are non-standard.\\
	\newline
	\textbf{Step 1.} Let $p(x) = x + x^2 + x^3 + x^4 + x^5 + x^6$, and explain why $(p(x))^2$ is the generating function for the probabilities of outcomes in rolling a pair of standard dice.\\
	\newline
	\textbf{Step 2.} Let $A = (a_1,a_2,a_3,a_4,a_5,a_6)$ and $B = (b_1,b_2,b_3,b_4,b_5,b_6)$ be two lists of positive integers.  Put $p_A(x) = x^{a_1} + x^{a_2} + x^{a_3} + x^{a_4} + x^{a_5} + x^{a_6}$ and $p_B(x) = x^{b_1} + x^{b_2} + x^{b_3} + x^{b_4} + x^{b_5} + x^{b_6}$. Explain why finding $a_i$s and $b_i$s such that $p_A(x)p_B(x) = (p(x))^2$ is relevant this part of the Experience.\\
	\newline
	\textbf{Step 3.} Factor $p(x)$ into irreducible polynomials and use this factorization to help solve for the $a_i$s and $b_i$s.  Specifically, the factorization will force the form of $p_A(x)$ to be something like $p_1(x)^qp_2(x)^rp_3(x)^sp_4(x)^t$, where $0\leq q,r,s,t\leq2$ and $p_i(x)$, for $ 1\leq i\leq4$, is a factor of $p(x)$.  In your solution to this step, you must motivate why you take this step.\\
	\newline
	\textbf{Step 4.} Begin to reduce the possibilities for q,r,s, and t by using information from $p_A(1)$ and $p_A(0)$.  Note that, on one hand $p_A(1) = 1^{a_1} + 1^{a_2} + 1^{a_3} + 1^{a_4} + 1^{a_5} + 1^{a_6} = 6\ (since\ a_i > 0)$, and on the other hand we have $p_A(1) = p_1(1)^qp_2(1)^rp_3(1)^sp_4(1)^t$.  Similarly, there are two ways to view $p_A(0)$.\\
	\newline
	\textbf{Step 5.} List all possible ways to label a pair of dice so that the probabilities of obtaining the sums 2,3,4,5,6,7,8,9,10,11,12 are $\frac{1}{36},\frac{2}{36},\frac{3}{36},\frac{4}{36},\frac{5}{36},\frac{6}{36},\frac{5}{36},\frac{4}{36},\frac{3}{36},\frac{2}{36},\frac{1}{36}$, respectively.  One such way will be the standard way.  In your solution for this step, explain why you have proved that the labels you have found are the only possible ones that give the desired probabilities for roll-outcomes.\\
	\newline
	\textbf{Solution: }\\
	\newline
% ***** solution section *****
	\textbf{Step 1:}\\
	Say we have $ax^n$, where $n$ represents the number rolled on a standard die and $a$ represents the number of ways you can roll n, then the probability of rolling a standard die, $p(x)$, can be represented as $p(x) = x + x^2 + x^3 + x^4 + x^5 + x^6$.\\
	Now when we have two standard dice, $n$ represents the sum of the number when the dice are rolled, and $a$ still represents the number of ways you can roll $n$.  Now we have:\\
	\[x^2 + 2x^3 + 3x^4 + 4x^5 + 5x^6 + 6x^7 + 5x^8 + 4x^9 + 3x^{10} + 2x^{11} + x^{12} \]
	which is equal to:\\
	\[(x + x^2 + x^3 + x^4 + x^5 + x^6)(x + x^2 + x^3 + x^4 + x^5 + x^6) = p(x)p(x) = (p(x))^2 \]
	\newline
	\textbf{Step 2:}\\
	Assume there are two dice $A+B$ labeled with positive integers that give you the same probabilities as regular dice.  Then \[A = \{a_1, a_2, a_3, a_4, a_5, a_6 \}\]
	\[B = \{b_1, b_2, b_3, b_4, b_5, b_6 \}\]
	If we use the same method of notation as in step 1 then
	\[p_A(x) = x^{a_1} + x^{a_2} + x^{a_3} + x^{a_4} + x^{a_5} + x^{a_6}\]
	\[p_B(x) = x^{b_1} + x^{b_2} + x^{b_3} + x^{b_4} + x^{b_5} + x^{b_6} \]
	Since we are finding all the ways to label a pair of non-standard dice such that the probabilities of rolling the usual sums is the same (represented by $(p(x))^2$).  Then, we need to find $a_i$s and $b_i$s such that $p_A(x)p_B(x) = (p(x))^2$.\\
	\newline
	\textbf{Step 3:}\\
	To help solve for the $a_i$s and $b_i$s we can factor $p(x)$ into irreducible polynomials.
	\[p(x) = (x + x^2 + x^3 + x^4 + x^5 + x^6) = x(x+1)(x^2 + x + 1)(x^2 - x + 1) \]
	we can also see that\\
	\[(p(x))^2 = x^2(x+1)^2(x^2 + x + 1)^2(x^2 - x + 1)^2 \]
	This forces the form of $p_A(x)$ to be $x^q(x+1)^r(x^2+x+1)^s(x^2-x+1)^t$ where $0\leq q,r,s,t\leq2$.\\
	\newline
	\textbf{Step 4:}\\
	To reduce the possibilities for q,r,s, and t. Let's use the info $p_A(1)$ and $p_A(0)$.\\
	$p_A(1)$ will help restrict our possibilities for r and s because $p_A(1) = 1^{a_1} + 1^{a_1} + 1^{a_1} + 1^{a_1} + 1^{a_1} + 1^{a_1} = 6 = (1)^q(1+1)^r(1^2+1+1)^s+(1^2 - 1 + 1)^t = 1^q2^r3^s1^t$.  From this we see that r and s must equal 1.\\
	\newline
	Now $p_A(0) = 0 = 0^q1^r1^s1^t$, so q cannot equal 0.\\
	If we let q=2 then $x^2(x+1)(x^2+x+1)(x^2-x+1)^0 = x^52x^42x^3x^2$ which gives us the set of labels \{5,4,4,3,3,2\} However, we want labels whose sums are the same as two standard die and since $a_i,b_i > 0$ then the smallest number on $b_i$ could only be a 1 therefore making the smallest sum of $a_i+b_i = 3$, when we need it to be 2.\\
	So our only option is to let $a=1$. So we know that $q=1,\ r=1,\ and\ s=1$ so now to find t which can either be 0,1, or 2.\\
	\newline
	\textbf{Step 5:}\\
	If we let $t=0$ then $x(x+1)(x^2+x+1)(x^2-x+1)^0 = x^4 + x^3 + x^3 + x^2 + x^2 + x$ which gives us the labels \{4,3,3,2,2,1\}\\
	\newline
	If $t = 1$ then:\\
	\[x(x+1)(x^2+x+1)(x^2-x+1)^1 = x^6 + x^5 + x^4 + x^3 + x^2 + x\]
	which gives us the labels \{6,5,4,3,2,1\} which is a regular die.\\
	Finally if $t=2$ then
	\[x(x+1)(x^2+x+1)(x^2-x+1)^2 = x^8 + x^6 + x^5 + x^4 + x^3 + x \]
	which gives us the labels \{8,6,5,4,3,1\}.\\
	\newline
	So the labels \{4,3,3,2,2,1\} and \{8,6,5,4,3,1\} are the only non-standard labels with positive integers such that the probabilities of rolling the usual sums 2,3,...,12 are the same as standard dice.\\
	To verify this we can also see that
	\[(x^4 + x^3 + x^3 + x^2 + x^2 + x)(x^8 + x^6 + x^5 + x^4 + x^3 + x)\\= x^2 + 2x^3 + 3x^4 + 4x^5 + 5x^6 + 6x^7 + 5x^8 + 4x^9 + 3x^{10} + 2x^{11} + x^{12} \]
	
	
	
	

% ######## sub 5 ########
	\textbf{Sub-Experience Five: Eul-ing the GF Machine}\\
	Recall that a $partition\ of\ a\ positive\ integer\ n$ is a nondecreasing sequence $\vec{\lambda} = (\lambda_1,\lambda_2,...)$ with $\sum_{i\geq1}^{}\lambda_i = n$.\\
	If the sequence $\vec{\lambda}$ has $k$ entries, then we call it a $partition\ of\ n\ into\ k\ parts$, and we call the $\lambda_i$s the $parts$.\\
	The following problems may each be solved in many ways, but all may be solved by examining an appropriately constructed generating function (not necessarily the same for function for each problem).  Note that each problem requests a number, but a proof that the number you produce is correct should be given.\\
	Suppose $S = \{a_1,a_1,...,a_k\}$ is a set of positive integers sharing no divisor; that is, $a_1,a_2,...a_k$ are $relatively\ prime$.  The $\textbf{Frobenius number}$ of S is the largest number that cannot be expressed as a linear combination, with integer coefficients, of numbers from S.  If $a_1$ and $a_2$ are relatively prime, their Frobenius number can be computed easily; it is $a_1a_2 - a_1 - a_2$.  But if $a_1,a_2,$ and $a_3$ are relatively prime, their Frobenius number is not known in general; that is, there is no formula for it --- it must be computed $ad\ hoc$.\\
	\begin{enumerate}
		\item Determine the number of partitions of 20 into at most 5 parts.  In other terms count the number of $\vec{\lambda}$s with no more than 5 entries that sum to 20.  For example (1,19),(1,1,1,1,16),(5,5,5,5),(2,3,4,5,6), and (20) are partitions of 20 to be included in the count.  These are not: (1,1,1,1,1,15), (2,2,2,2,2,2,2,2,2,2), (3,3,3,3,3,5).
		\item Determine the number of partitions of 30 into parts of size less than or equal to 5.  Here, a partition of 30 is not counted if any $\lambda_i > 5$.  So these partitions of 30 are counted: (1,1,1,1,1,1,1,1,1,1,1,1,1,1,1,1,1,1,1,1,5,5), (1,1,2,2,2,3,4,5,5,5), (5,5,5,5,5,5); while these are not: (1,1,1,1,1,1,1,1,1,1,1,1,1,1,1,1,1,1,1,1,10), (6,6,6,6,6),
		\item Determine $p(20)$, the number of partitions of 20.
		\item Determine the Frobenius number for the set \{7,9,11\}.\\
	\end{enumerate}
	\textbf{Solution: }\\
	\newline
% ***** solution section *****
\textbf{Part 2:}

\[\left( {{x}^{28}}+{{x}^{24}}+{{x}^{20}}+{{x}^{16}}+{{x}^{12}}+{{x}^{8}}+{{x}^{4}}+1\right) \cdot \left( {{x}^{30}}+{{x}^{25}}+{{x}^{20}}+{{x}^{15}}+{{x}^{10}}+{{x}^{5}}+1\right) \cdot \left( {{x}^{30}}+{{x}^{27}}+{{x}^{24}}+{{x}^{21}}+{{x}^{18}}+{{x}^{15}}+{{x}^{12}}+{{x}^{9}}+{{x}^{6}}+{{x}^{3}}+1\right) \cdot \left( {{x}^{30}}+{{x}^{28}}+{{x}^{26}}+{{x}^{24}}+{{x}^{22}}+{{x}^{20}}+{{x}^{18}}+{{x}^{16}}+{{x}^{14}}+{{x}^{12}}+{{x}^{10}}+{{x}^{8}}+{{x}^{6}}+{{x}^{4}}+{{x}^{2}}+1\right) \cdot \left( {{x}^{30}}+{{x}^{29}}+{{x}^{28}}+{{x}^{27}}+{{x}^{26}}+{{x}^{25}}+{{x}^{24}}+{{x}^{23}}+{{x}^{22}}+{{x}^{21}}+{{x}^{20}}+{{x}^{19}}+{{x}^{18}}+{{x}^{17}}+{{x}^{16}}+{{x}^{15}}+{{x}^{14}}+{{x}^{13}}+{{x}^{12}}+{{x}^{11}}+{{x}^{10}}+{{x}^{9}}+{{x}^{8}}+{{x}^{7}}+{{x}^{6}}+{{x}^{5}}+{{x}^{4}}+{{x}^{3}}+{{x}^{2}}+x+1\right) \]

\[...+1006\cdot {{x}^{34}}+913\cdot {{x}^{33}}+828\cdot {{x}^{32}}+747\cdot {{x}^{31}}+674\cdot {{x}^{30}}+603\cdot {{x}^{29}}+540\cdot {{x}^{28}}+480\cdot {{x}^{27}}+427\cdot {{x}^{26}}+377\cdot {{x}^{25}}+333\cdot {{x}^{24}}+291\cdot {{x}^{23}}+255\cdot {{x}^{22}}+221\cdot {{x}^{21}}+192\cdot {{x}^{20}}+...\]

\textbf{Part 3:}
\tiny\[\left( {{x}^{11}}+1\right) \cdot \left( {{x}^{12}}+1\right) \cdot \left( {{x}^{13}}+1\right) \cdot \left( {{x}^{14}}+1\right) \cdot \left( {{x}^{14}}+{{x}^{7}}+1\right) \cdot \left( {{x}^{15}}+1\right) \cdot \left( {{x}^{16}}+1\right) \cdot \left( {{x}^{16}}+{{x}^{8}}+1\right) \]\[\cdot \left( {{x}^{17}}+1\right) \cdot \left( {{x}^{18}}+1\right) \cdot \left( {{x}^{18}}+{{x}^{9}}+1\right) \cdot \left( {{x}^{18}}+{{x}^{12}}+{{x}^{6}}+1\right) \cdot \left( {{x}^{18}}+{{x}^{15}}+{{x}^{12}}+{{x}^{9}}+{{x}^{6}}+{{x}^{3}}+1\right) \cdot \left( {{x}^{19}}+1\right) \]\[\cdot \left( {{x}^{20}}+1\right) \cdot \left( {{x}^{20}}+{{x}^{10}}+1\right) \cdot \left( {{x}^{20}}+{{x}^{15}}+{{x}^{10}}+{{x}^{5}}+1\right) \cdot \left( {{x}^{20}}+{{x}^{16}}+{{x}^{12}}+{{x}^{8}}+{{x}^{4}}+1\right)\]\[ \cdot \left( {{x}^{20}}+{{x}^{18}}+{{x}^{16}}+{{x}^{14}}+{{x}^{12}}+{{x}^{10}}+{{x}^{8}}+{{x}^{6}}+{{x}^{4}}+{{x}^{2}}+1\right) \]\[\cdot \left( {{x}^{20}}+{{x}^{19}}+{{x}^{18}}+{{x}^{17}}+{{x}^{16}}+{{x}^{15}}+{{x}^{14}}+{{x}^{13}}+{{x}^{12}}+{{x}^{11}}+{{x}^{10}}+{{x}^{9}}+{{x}^{8}}+{{x}^{7}}+{{x}^{6}}+{{x}^{5}}+{{x}^{4}}+{{x}^{3}}+{{x}^{2}}+x+1\right) \]

\normalsize\[\mathrm{\tt (\%o18) }\quad ...+627\cdot {{x}^{20}}+490\cdot {{x}^{19}}+385\cdot {{x}^{18}}+297\cdot {{x}^{17}}+231\cdot {{x}^{16}}+176\cdot {{x}^{15}}+135\cdot {{x}^{14}}+101\cdot {{x}^{13}}+\]\[77\cdot {{x}^{12}}+56\cdot {{x}^{11}}+42\cdot {{x}^{10}}+30\cdot {{x}^{9}}+22\cdot {{x}^{8}}+15\cdot {{x}^{7}}+11\cdot {{x}^{6}}+7\cdot {{x}^{5}}+5\cdot {{x}^{4}}+3\cdot {{x}^{3}}+2\cdot {{x}^{2}}+x+1\]


% ######## sub 6 ########
	\textbf{Sub-Experience Six: The Twelve-Fold Way}\\
	Suppose you have a set N of $n$ objects which may be labeled or not.  You also have a set $X$ of $x$ containers which may be labeled or not.  Please determine formula for the number of distributions of the objects in N into the containers of X as the possibilities range over N's objects labeled or not, and X's containers labeled or not, and where the following restrictions on containers hold:  (1) no restrictions,  (2) each container may hold at most one object, and lastly (3) no container may be empty.  The table below catalogs all the possibilities.  So the task is to determine a formula which is the number of distributions of N into X satisfying the corresponding entry's constraints.  Another perspective on this problem is as follows.  Count the number of distinct functions f:N $\rightarrow X$; the restrictions on the containers translate correspondingly to (1) no restriction on f, (2) f is one-to-one, (3) f is onto.\\
	\newline
	\textbf{Twelve Formula}\\
	\begin{center}
		\begin{tabular}{c|c||c|c|c}
			\hline
			N&X&unrestricted&at most one&nonempty\\
			\hline
			labeled&labeled&1.&2.&3.\\
			unlabeled&labeled&4.&5.&6.\\
			labeled&unlabeled&7.&8.&9.\\
			unlabeled&unlabeled&10.&11.&12.\\
			\hline
			\end{tabular}
	\end{center}
	Please: No resources, outside of class notes and the help of others in class, are allowed on this part of the experience.\\
	\newline
	\textbf{Solution: }\\
	\newline
% ***** solution section *****

\begin{enumerate}
	\item Let $N=\{1, 2, ..., n\}$, where each member of $N$ represents a ball.  Let $X=\{1, 2, ..., x\}$, where each member of $X$ represents a container where the balls of $N$ can be contained.  Because there are no restrictions, each member of $N$ can go into each member of $X$ without affecting the placement of any other member of $N$.  Because there are $x$ places where a member of $N$ can go \textbf{HOLE}, the number of ways that the balls in $N$ can be placed in the containers of $X$ is $x^{n}$.
	
	\item Now remembering the definitions of $N$, $X$, $n$, and $x$, we are now restricting the members of $X$ such that there can be at most one member of $N$ in a member of $X$.  When starting to place labeled members of $N$ in members of $X$, the first member of $N$ has $x$ possible containers in which it can be placed.  When trying to place the second member of $N$, there are now only $x-1$ places available because one of the members of $X$ has already been filled to the level of restriction.  Following the same pattern, the third member of $N$ only has $x-2$ possibilities, the fourth has $x-3$ possibilities, and the $n$th has $x-(n-1)$ possibilities.  Therefore the the number of ways that the balls of $N$ can be placed in the containers of $X$ is $x^{\underline{n}}$.
	
	\item The restriction on the labeled sets of $N$ and $X$ is that now there must be at least one member of $N$ in each member of $X$.  First you need to find the number of ways that the members of $N$ can be distributed to the members of $X$.  In order to do that you find $\stirling{n}{k}$.  Now that you have that you need to apply that number to all the members of $X$, with the result being $x!\stirling{n}{k}$.
\end{enumerate}

\end{document}