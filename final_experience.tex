\documentclass[10pt,a4paper]{report}
\usepackage[letterpaper, portrait, margin=1in]{geometry}
\usepackage[latin1]{inputenc}
\usepackage{amsmath}
\usepackage{amsfonts}
\usepackage{amssymb}
\usepackage{graphicx}
\author{Seth Bertlshofer\\Alexis Tyler\\Dustin Ginos\\Kevin Burgon}
\title{Desc. Experience 1}
\graphicspath{{img/}}
\begin{document}
	\maketitle
	\textbf{Sub-Experience One: The Mad Mail Carrier}\\
	Suppose Sue is a Mail Carrier who is crazy.  He likes to ensure that non of the $n$ houses on his delivery route get the mail they are supposed to.  Your goal, should you choose to accept it, for this sub-experience is to determine the number of ways Sue can deliver mail so that no one gets their mail in two ways.  One method is to use the Principle of Inclusion/Exclusion (PIE).  The other mothod is to use an exponential generating function to solve a recurrence which you'll develop.  Put $D_n$ equal to the number of ways Sue can distribute mail to $n$ houses so that none of them gets the correct mail.
	
	\textbf{Sub-Experience One, Part One: PIE Approach}
	Use the PIE to determine $D_n$.\\
	
	\textbf{Solution: }\\
	\newline
	
	\textbf{Sub-Experience Two, Part One: PIE Approach}
	The formula you obtain above should involve a truncated power series for $e^{-1}$.  Show that $D_n = \lfloor\frac{n!}{e}+\frac{1}{2}\rfloor, for n > 0$. (For $n=0$, the formula doesn't work: $D_0 = 1$, but the formula gives 0.)\\
	\textbf{Solution: }\\
	\newline
	
	\textbf{Sub-Experience Three, Part One: PIE Approach}
	\textbf{SE 1.3.1.} Prove the recurrence $D_n = (n-1)D_{n-1} + (n-1)D_{n-2},\ for\ n\geq 2,\ and\ D_0 = 1, D_1=0$.\\
	\textbf{SE 1.3.2.} Deduce, from the above recurrence $D_n = nD_{n-1}+(-1)^n,\ for\ n\geq 1,\ and\ D_0 = 1$.\\
	\textbf{SE 1.3.3.} Use an exponential generating function to solve the recurrence from part 1.3.2\\
	\textbf{Solution: }\\
	\newline
	
	
	\textbf{Sub-Experience Two: Stirling Numbers of the Second Kind}\\
	Recall that a partition of a set $X$ into $k$ blocks is a set $\prod = \{B_1,...,B_k\}$ where the $B_i$s are disjoint nonempty subsets of $X$ whose union is $X$. Define ${n}\choose{k}$ to be the number of partitions of a set with $n$ elements into $k$ blocks.\\
	\textbf{2.1} Use the Principle of Inclusion/Exclusion to create a formula for ${n}\choose{k}$.  Please verify that the formula you find actually works.\\
	\textbf{2.2} Use exponential generating functions to create a formula for ${n}\choose{k}$.  Please verify that the formula you find actually works.  It may appear different than the found in part 2.1, so be careful.\\
	\textbf{Solution: }\\
	\newline
	\textbf{Sub-Experience Three: Catalan Numbers}\\
	Roll up your sleeves and prepare to get down and dirty with generating functions.  The goal of the following sequence of problems is to prove that the sequence $(b_n)_n\geq0$ generated by the recurrence relation\\
	\[b_n = b_0b_{n-1} + b_1b_{n-2} + ... + b_{n-1}b_0\ with b_0 = 1\]
	gives the answer to the following counting questions, and that $b_n = \frac{1}{n+1}{2n}\choose{n}$.  These numbers are called the $Catalan\ Numbers$ and are nearly as ubiquitous as the Fibonacci Numbers.\\
	\newline
	How many binary trees on $n$ vertices are there?  Follow the steps below to produce a closed formula for $b_n$ using a generating function.\\
	\newline
	3.0 Denote by $b_n$ the number of binary trees on vertices.  Please prove that $b_n = b_0b_{n-1} + b_1b_{n-2} + ... + b_{n-1}b_0$ with $b_0 = 1$.\\
	3.1 Let $g(x) = \sum_{n=0}^{\infty}b_nx^n$ be the generating function for the sequence $b_n n\geq0$ of the number of binary trees on $n$ vertices.  Show that $g(x) = 1+x(g(x))^2$.\\
	3.2 Use the quadratic formula to show that $g(x) = 1-\frac{\sqrt{1-4x}}{2x}$. Note that you will have two options to use for $g(x)$, one of which you must rule out.\\
	3.3 Recall $The\ Binomial\ Theorem:\ For\ any\ real\ number\ r,\ (1+z)^r = \sum_{k=0}^{\infty}\binom{r}{k}z^k$, where 
	\[\binom{r}{k} = \lbrace \binom{\frac{r(r-1)(r-2)...(r-k+1)}{k!}}{0,}\binom{r\in\mathbb{R},0\leq k\in\mathbb{Z}}{k<0}\]
	Use the Binomial Theorem to obtain $g(x) = \frac{1}{2x}(1-\sum_{k=0}^{\infty}\binom{\frac{1}{2}}{k}(-4x)^k)$.\\
	3.4 Manipulate $g(x) = \frac{1}{2x}(1-\sum_{k=0}^{\infty}\binom{\frac{1}{2}}{k}(-4x)^k)$ into $g(x) = \sum_{n=0}^{\infty}\binom{\frac{1}{2}}{k}(-1)^n2^{2n+1}x^n$.  The reason for doing this is so that $x^n$'s coefficient can be read (remember that $b_n$ is defined to be$ [x^n]g(x)$.)  The following steps are suggested, but not required; follow them only if you have no idea of your own.  Begin with $g(x) = \frac{1}{2x}(1-\sum_{k=0}^{\infty}\binom{\frac{1}{2}}{k}(-4x)^k)$.\\
	\begin{itemize}
		\item Pull out the first term on the summation,  and simplify
		\item Substitute $n+1$ for k;  This will allow the summation to begin at zero.
		\item Pull the negative that resulted from the first manipulation in this list into the summation. 
		\item Bring $\frac{1}{2x}$ into the summation.
	\end{itemize}
	
	3.5 Verify that $\binom{\frac{1}{2}}{n+1} = \frac{1}{n+1}\binom{\frac{n-1}{2}}{n}(-1)^n\frac{1}{2}$\\
	3.6 Verify that $\binom{2n}{n}\frac{1}{2^{2n}} = \binom{\frac{n-1}{2}}{n}$.\\
	3.7 Finally, coerce $g(x)$ into $\sum_{n\geq0}^{}\binom{2n}{n}\frac{x^n}{n+1}$.\\
	\textbf{Solution: }\\
	\newline
	\textbf{Sub-Experience Four: Non-Standard Dice}\\
	A 6-sided die labeled with the integers 1,2,3,4,5,6 will be called a $standard\ die$.  The goal for this part of the Midterm Experience is to determine all ways to label a pair of dice with positive integers so that the probabilities of rolling the usual sums 2,3,...,12 are the same, but the labels are non-standard.\\
	\newline
	\textbf{Step 1.} Let $p(x) = x + x^2 + x^3 + x^4 + x^5 + x^6$, and explain why $(p(x))^2$ is the generating function for the probabilities of outcomes in rolling a pair of standard dice.\\
	\newline
	\textbf{Step 2.} Let $A = (a_1,a_2,a_3,a_4,a_5,a_6)$ and $B = (b_1,b_2,b_3,b_4,b_5,b_6)$ be two lists of positive integers.  Put $p_A(x) = x^{a_1} + x^{a_2} + x^{a_3} + x^{a_4} + x^{a_5} + x^{a_6}$ and $p_B(x) = x^{b_1} + x^{b_2} + x^{b_3} + x^{b_4} + x^{b_5} + x^{b_6}$. Explain why finding $a_i$s and $b_i$s such that $p_A(x)p_B(x) = (p(x))^2$ is relevant this part of the Experience.\\
	\newline
	\textbf{Step 3.} Factor $p(x)$ into irreducible polynomials and use this factorization to help solve for the $a_i$s and $b_i$s.  Specifically, the factorization will force the form of $p_A(x)$ to be something like $p_1(x)^qp_2(x)^rp_3(x)^sp_4(x)^t$, where $0\leq q$, r,s,$t\leq2$ and $p_i(x)$, for $ 1\leq i\leq4$, is a factor of $p(x)$.  In your solution to this step, you must motivate why you take this step.\\
	\newline
	\textbf{Step 4.} Begin to reduce the possibilities for q,r,s, and t by using information from $p_A(1)$ and $p_A(0)$.  Note that, on one hand $p_A(1) = 1^{a_1)} + 1^{a_2)} + 1^{a_3)} + 1^{a_4)} + 1^{a_5)} + 1^{a_6)} = 6(since\ a_i > 0)$, and on the other hand we have $p_A(1) = p_1(1)^qp_2(1)^rp_3(1)^sp_4(1)^t$.  Similarly, there are two ways to view $p_A(0)$.\\
	\newline
	\textbf{Step 5.} List all possible ways to label a pair of dice so that the probabilities of obtaining the sums 2,3,4,5,6,7,8,9,10,11,12 are $\frac{1}{36},\frac{2}{36},\frac{3}{36},\frac{4}{36},\frac{5}{36},\frac{6}{36},\frac{5}{36},\frac{4}{36},\frac{3}{36},\frac{2}{36},\frac{1}{36}$, respectively.  One such way will be the standard way.  In your solution for this step, explain why you have proved that the labels you have found are the only possible ones that give the desired probabilities for roll-outcomes.\\
	\newline
	\textbf{Solution: }\\
	\newline
	\textbf{Sub-Experience Five: Eul-ing the GF Machine}\\
	\textbf{Solution: }\\
	\newline
	\textbf{Sub-Experience Six: The Twelve-Fold Way}\\
	\textbf{Solution: }\\
	\newline
	\textbf{[Bonus] Sub-Experience Seven:}\\	
	\textbf{Solution: }\\
	\newline
\end{document}